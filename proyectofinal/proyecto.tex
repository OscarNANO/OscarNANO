\documentclass[10pt,twocolumn]{article}

\usepackage[spanish]{babel}
\usepackage[numbers,sort&compress]{natbib}
\usepackage[T1]{fontenc}
\usepackage[ansinew]{inputenc}
\usepackage{graphicx}
\usepackage{url}
\usepackage{subcaption}
\usepackage{caption}
\usepackage{listings}
\usepackage{enumerate}
\usepackage{amsmath}
\usepackage{float}
\usepackage[numbers,sort&compress]{natbib}

\begin{document}

\twocolumn[
  \begin{@twocolumnfalse}

\title{\textbf{An\'alisis Metalogr\'afico}}
\author{Oscar Qui\~nonez}

  \maketitle
    \begin{abstract}
      \begin{center}
      Lugar para escribir el resumen
 
      \end{center}
    \end{abstract}
    Palabras clave: \textit{keywords.\\ \\}
  \end{@twocolumnfalse}
  ]


\section{Introducci\'on}\label{intro}
 
Aqu\'i se mencionar\'a que es un an\'alisis metalogr\'afico y para que nos sirve.


\section{Antecedentes}\label{antes}

Trabajos realizados anteriormente \cite{satuelisa}


\section{Trabajos relacionados}\label{trabajos}

Trabajos actuales \cite{doctora}

 

\section{Metodolog\'{i}a}\label{met}

Aqu\'i se escribe a detalle lo que se intenta representar en la simulaci\'on



\section{Resultados y Discusi\'{o}n}\label{res}

Imagenes, tablas, graficas, etc. que nos haya dado el programa al correr el c\'odigo.


\section{Conclusi\'{o}n}\label{con}

Deducciones que nos deja el trabajo realizado y su interpretaci\'on \cite{yo} 


  \bibliography{proyecto}
  \bibliographystyle{unsrtnat}
\end{document}